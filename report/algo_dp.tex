\documentclass[../report.tex]{subfiles}
%\usepackage[left=2cm, right=2cm, top=2.5cm]{geometry}
%\usepackage{algorithm}
%\usepackage{algorithmic}

\begin{document}
The dynamic programming implementation is based on the Bellman-Held-Karp algorithm  proposed in 1962 independently by Bellman[1] and by Held and Karp.
\newline{}
Dynamic programming express a solution for a problem through the solution of smaller problem.
\newline{}
Let's consider a set of node $S=\{0,1,......,n\}$. We want to compute the minimum cost for a path starting at node 0 and visiting all nodes exactly once.
\newline{}
\paragraph{Optimal solution}\hfill \break
Let's consider a set S' $\subseteq$ S and $C(S',j)$ the minimum cost for a path between node 0 and node j containing all nodes from S'. Then the cost $C(S',j)$ can be decomposed in the sum of the minimum cost of the path from node 0 to k including all nodes from $S'-\{j\}$, and the distance $d_{kj}$.

$$C(S',j)=min_{k \in S'-\{j\}, k \ne j}\{C(S',k)+d_{kj}\}$$

\paragraph{Base case}\hfill \break
If $S'=\{0\}$, $$C(S',0)=0$$.

\paragraph{Pseudo code}\hfill

\begin{center}
	\colorbox[gray]{0.95}{
	\begin{minipage}{0.65\textwidth}
			\begin{algorithm}[H]
			\caption{Dynamic Programming Algorithm}
			\begin{algorithmic} 
			\STATE Initialization
			\FOR {j := 2 to n do}
			\STATE $C(\{j\},j)=0$
			\ENDFOR

			\FOR {$subsetsize$ := 2 to n-1}
			   \FOR {all $ S' \subseteq \{2,...,n \}$ with $|S'|=subsetsize$}
				  \FOR {all j in S'}
					 \STATE $C(S',j)=min_{k \in S'-\{j\}}\{C(S'-\{j\},k)+d_{kj}\}$
				  \ENDFOR
			   \ENDFOR
			\ENDFOR
			\STATE $mincost := min_{1<j \leq n}\{C(\{2,...,n\},j)+d_{j0}\}$
			\STATE return $mincost$
			\end{algorithmic}
			\end{algorithm}
	 \end{minipage}}
\end{center}

\paragraph{Implementation}\hfill \break
We use a bit field to code the subset of nodes selected in the set \{1,...n\}.
Each node j is code by the value $2^j$ when selected and 0 when not selected. 

\paragraph{Complexity}\hfill \break
   We need to build all the subsets of \{1,...,n\} i.e. $2^n$ subsets. For all nodes j of each subset (at most n nodes) and for all nodes k distinct from node j(at most n-1 nodes), we compute the cost of the sub paths terminated by node k for the subset deprived of node j.
Thus the time complexity is $2^n n^2$.
\newline{} In term of space complexity, if we use a bit field to code all subsets for each node $j \subseteq \{0,...,n\}$, the required space is $2^n (n+1)$.
\newline{} Nevertheless, we can note that to compute the cost for the subsets of size s, we only need the cost of the subsets of size (s-1). Thus the space complexity can be reduced to 


\end{document}


